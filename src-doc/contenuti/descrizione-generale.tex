\chapter{Descrizione generale}

\section{Obiettivi del prodotto}
L'obiettivo del progetto è la realizzazione di un portale, che permette la visualizzazione, la gestione e la manutenzione di entità varie, tra cui lampioni, sensori, assieme ad una gestione del ticketing.
Nello specifico, lo scopo finale del progetto è quello di fornire agli utenti la capacità di interagire con la piattaforma con limiti definiti per tipologia di utente, con il supporto di entità automatizzate.

\section{Caratteristiche degli utenti}

Presupponiamo che più tipologie di utenti fisici interagiscano con il sistema, questo richiede un sistema di autenticazione e di autorizzazione.
Ognuno dei nostri utenti che analizzeremo avrà specifici ruoli, bisogni e responsabilità.
Il loro mezzo di interazione è un'interfaccia web\footnote{come richiesto dal capitolato}, tramite questa potranno accedere alle funzionalità di cui hanno bisogno.

Descriviamo ora le tipologie di utenti e i loro bisogni.

\paragraph{Utente gestore:} L'utente gestore avrà bisogno di visualizzare i lampioni e il loro stato, di modificarne l'intensità luminosa e di settare vari parametri sugli stessi. Dovrà inoltre poter segnalare la presenza di guasti all'utente manutentore.

\paragraph{Utente manutentore:} L'utente manutentore si occupa della manutenzione del sistema fisico e svolge compiti vari, come sostituire lampadine o installare nuovi lampioni.

\paragraph{Utente non autenticato:} L'utente non autenticato è quell'attore che non ha ancora compiuto il processo di autenticazione. Compiuta l'autenticazione, il sistema lo riconoscerà in uno dei casi precedenti oppure segnalerà il problema.

\section{Descrizione attori}
\paragraph{Utente non autenticato} 
Persona non ancora connessa al sistema, può solo autenticarsi;

\paragraph{Utente gestore} 
Persona che gestisce il sistema, può visualizzare tutte le informazioni disponibili, gestire le impostazioni e aprire ticket;

\paragraph{Utente manutentore} 
Persona che si occupa di riparazioni e della gestione degli apparecchi. Può aggiungere, rimuovere e sostituire lampioni e sensori, chiudere ticket;

\paragraph{Sensore di presenza} 
Strumento che rileva la presenza di persone in un'area e regola la luminosità dei lampioni;

\paragraph{Time condition}
Condizione imposta per l'abbassamento automatico della luminosità;

\paragraph{Sensore di stato} 
Strumento per rilevare automaticamente i guasti, può aprire ticket di guasto;

\paragraph{Lampione} 
Oggetto utilizzato per l'emissione della luce;

\paragraph{Coordination service} 
Sistema che si occupa di coordinare le varie funzioni di aggiunta e gestione;

\paragraph{Auth service} 
Sistema che si occupa di controllare le credenziali di un utente che si vuole autenticare.

\section{Vincoli progettuali}

Il progetto pone alcuni vincoli preventivi su consegna, tecnologie e prestazioni.

\subsection{Vincoli di consegna}
\begin{itemize}
    \item Copertura di test $\geq 80\%$ correlata di report;
    \item Documentazione di scelte implementative e progettuali;
    \item Documentazione di problemi aperti ed eventuali soluzioni da esplorare.
\end{itemize}

\subsection{Vincoli tecnlogici}
\begin{itemize}
    \item Un'interfaccia utente esposta tramite un'applicazione web responsive che possa funzionare su dispositivi android o iOs muniti di browser web;
    \item Utilizzo di sistemi embedded\footnote{Probabilmente \href{https://www.raspberrypi.com/products/raspberry-pi-zero-w/}{raspberry pi zero}} che non utilizzono tecnologie esterne al sistema, si richiede un'implementazione specifica controllo hardware dei lampioni;
    \item Utilizzo di un'architettura a microservizi.
\end{itemize}

\subsubsection{Tecnologie consigliate}

\begin{itemize}
    \item Utilizzo di Java per il sistema di coordinamento generale.
\end{itemize}

\subsection{Vincoli prestazionali}

Il sistema suppone di utilizzare istanze da massimo 2 CPU e 1GB di memoria.

Si richiede poi, tenendo presente quanto sopra detto, un'analisi di carico massimo e di servizi cloud più adatti a supportare il sistema.

\subsection{Vincoli aggiuntivi e opzionali}
\begin{itemize}
    \item Cifratura di tutte le comunicazioni tra app e server per garantire validità di informazioni.
\end{itemize}
