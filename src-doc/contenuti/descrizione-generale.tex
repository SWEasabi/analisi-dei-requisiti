\chapter{Descrizione generale}

\section{Vincoli progettuali}

Il progetto pone alcuni vincoli preventivi su consegna, tecnologie e prestazioni.

\subsection{Vincoli di consegna}
\begin{itemize}
    \item Copertura di test $\geq 80\%$ correlata di report;
    \item Documentazione di scelte implementative e progettuali;
    \item Documentazione di problemi aperti ed eventuali soluzioni da esplorare.
\end{itemize}



\subsection{Vincoli tecnlogici}
\begin{itemize}
    \item Un'interfaccia utente esposta tramite un'applicazione web responsive che possa funzionare su dispositivi android o iOs muniti di browser web;
    \item Utilizzo di sistemi embedded\footnote{Probabilmente \href{https://www.raspberrypi.com/products/raspberry-pi-zero-w/}{raspberry pi zero}} che non utilizzono tecnologie esterne al sistema, si richiede un'implementazione specifica controllo hardware dei lampioni;
    \item Utilizzo di un'architettura a microservizi.
\end{itemize}

\subsubsection{Tecnologie consigliate}

\begin{itemize}
    \item Utilizzo di Java per il sistema di coordinamento generale.
\end{itemize}

\subsection{vincoli prestazionali}

Il sistema suppone di utilizzare istanze da massimo 2 CPU e 1Gb di memoria.

Si richiede poi, tenendo presente quanto sopra detto, un'analisi di carico massimo e di servizi cloud più adatti a supportare il sistema.

\subsection{vincoli aggiuntivi e opzionali}
\begin{itemize}
    \item Cifratura di tutte le comunicazioni tra app e server per garantire validità di informazioni.
\end{itemize}

