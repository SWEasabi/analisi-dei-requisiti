\chapter{Casi d'uso}\label{casi-d-uso}
\section{Attori}

Si richiama alla sezione relativa, \ref{attori} nel capitolo relativo alle user stories.

\section{Sistema di autenticazione}

Il sistema di autenticazione si occupa di gestire gli account degli utenti, di autenticarli e di autorizzarli a compiere determinate azioni.

\paragraph{Obiettivo} Il sistema di autenticazione si propone come goal principale quello di offrire all'utente non autenticato la possibilità di autenticarsi, questa condizione viene poi utilizzata come precondizione in molti degli altri sistemi e casi d'uso.

\begin{figure}[h]
    \includegraphics[width=\textwidth]{contenuti/img/casi_uso_grafici-applicazione,autenticazione.png}
    \caption{Parte relativa all'autenticazione dell'applicazione}
    \label{fig:autenticazione}
\end{figure}

\subsection{I casi d'uso descritti}

\begin{itemize}
    \item UC01
    \item UC01.1
\end{itemize}

\section{Sistema di gestione}

\begin{figure}[H]
    \includegraphics[width=\textwidth]{contenuti/img/casi_uso_grafici-applicazione,gestione.png}
    \caption{Parte relativa alla gestione dell'applicazione}
    \label{fig:gestione}
\end{figure}

\subsection{I casi d'uso descritti}

\begin{itemize}
    \item UC02 Visualizzazione lista aree
    \item UC03 Visualizzazione dettaglio area
    \item UC04 Visualizzazione dettaglio sensore
    \item UC05 Visualizzazione dettaglio lampione
    \item UC06 Apertura ticket di guasto
    \item UC07 Chiusura ticket
    \item UC08 Aggiunta area al sistema
    \item UC09 Inserimento sensore a sistema
    \item UC10 Inserimento lampione a sistema
    \item UC11 Rimozione area dal sistema
    \item UC12 Rimozione sensore dal sistema
    \item UC13 Rimozione lampione dal sistema
\end{itemize}

\section{Sistema di coordinamento}



%lista use-cases
\section{UC01 - Login}

\paragraph{Descrizione:}
Come utente non loggato, voglio autenticarmi utilizzando le mie username e password, così da poter accedere al sistema.

\paragraph{Attore principale:}
Utente non loggato

\paragraph{Precondizioni:}
Il sistema non riconosce l'utente.

\paragraph{Post-condizioni:}
Il sistema riconosce l'utente.

\paragraph{Scenario principale:}
\begin{enumerate}
    \item L'utente inserisce il suo username;
    \item L'utente inserisce la sua password.
\end{enumerate}

\paragraph{Scenario alternativo:}
\begin{enumerate}
    \item L'utente fornisce delle credenziali errate e il sistema ritorna un errore.
\end{enumerate}

\section{UC02 - Visualizzazione lista aree}\label{uc:02}

\paragraph{Intenzione in contesto} L'attore primario vuole vedere una lista delle varie aree di illuminazione ed il loro stato.

\paragraph{Attore primario} L'attore primario sono l'utente gestore e manutentore.

\paragraph{Precondizioni} L'attore primario è autenticato ed autorizzato dal sistema.

\paragraph{Postcondizioni} L'attore primario vede la lista delle aree.

\paragraph{Scenario principale}

\begin{enumerate}
    \item L'utente richiede la visualizzazione della lista delle aree;
    \item il sistema fornisce la lista delle aree presenti nel sistema.
\end{enumerate}

\begin{figure}[h]
    \includegraphics[width=\textwidth]{contenuti/img/casi_uso_grafici-uc2.png}
    \caption{Dettaglio dell'UC02}
    \label{fig:uc02}
\end{figure}

\subsection{UC02.1 Visualizzazione lampioni in area}
\paragraph{Intenzione in contesto} L'attore primario vuole visualizzare una singola area.

\paragraph{Attore primario} L'attore primario è l'utente gestore.
\paragraph{Precondizioni}L'attore primario è riconosciuto ed autorizzato dal sistema.
\paragraph{Postcondizioni} L'attore primario visualizza la singola area desiderata.

\paragraph{Scenario principale}
\begin{enumerate}
    \item L'utente richiede di visualizzare una singola area;
    \item l'utente visualizza la singola area desiderata.
\end{enumerate}

\section{UC03 - regolare luminosità molteplici lampioni}

\paragraph{Descrizione:}
Come utente loggato, voglio poter manipolare lo stato di molteplici lampioni.

\paragraph{Attore principale:}

Utente Gestore

\paragraph{Precondizioni:}
L'utente deve essere loggato e deve essere utente gestore, dalla percentuale impostata a tempo 0.

\paragraph{Post-condizioni:}
I lampioni sono manipolati nella intensità scelta.
\paragraph{Scenario principale:}
\begin{itemize}
    \item utente gestore manipola lo slider dell'intensità luminosa;
    \item utente gestore preme lo stato di conferma;
\end{itemize}

\section{UC04 - Ticket di guasto}

\paragraph{Descrizione:} come utente gestore desidero aprire un ticket di assistenza ogni qual volta si verifichi un guasto ad un impianto di illuminazione:
\begin{itemize}
    \item il sistema rileva ogni
    eventuale malfunzionamento dell’impianto di illuminazione
    \item un utente gestore loggato inserisce un nuovo guasto ad un impianto di illuminazione
\end{itemize}


\paragraph{Attore principale:}
Utente gestore.

\paragraph{Precondizioni:}
Il sistema e/o l'utente gestore devono segnalare un avvenuto guasto ad un impianto di illuminazione

\paragraph{Post-condizioni:}
Il ticket di guasto è stato aggiunto.

\paragraph{Scenario principale:}
\begin{enumerate}
    \item Il sistema rileva o l'utente gestore inserisce un malfunzionamento ad un impianto di illuminazione;
    \item Il sistema informa il gestore sul guasto e apre un ticket di assistenza;
    \item La piattaforma di ticketing identifica la località del guasto;
    \item Il guasto viene risolto e l'impianto ritorna funzionante
\end{enumerate}

\paragraph{Scenario alternativo:}
\begin{enumerate}
    \item L'utente gestore fornisce un'area geografica per l'impianto malfunzionanete inesistente;
    \item Il guasto rende l'impianto parzialmente o totalmente inutilizzabile rendendo quindi necessaria una sostituzione;
\end{enumerate}

\section{UC05 - Visualizzazione dettaglio lampione}\label{uc:05}
\paragraph{Attore primario} L'attore primario è l'utente gestore.
\paragraph{Intenzione in contesto} L'attore primario vuole vedere i dettagli di un lampione, di questo vuole conoscerne lo stato.
\paragraph{Precondizioni}L'attore primario è riconosciuto ed autorizzato dal sistema.
\paragraph{Postcondizioni} L'attore primario visualizza i dettagli e le informazioni su uno specifico lampione.
\paragraph{Scenario principale}
\begin{enumerate}
    \item L'utente richiede di visualizzare i dettagli di uno specifico lampione;
    \item l'utente visualizza i dettagli dello specifico lampione.
\end{enumerate}
\section{UC06 - Apertura ticket di guasto}\label{uc:06}

\paragraph{Intenzione in contesto} L'attore primario deve aprire un ticket per segnalare un guasto al sistema.

\paragraph{Attore primario} L'attore primario sono o l'utente gestore oppure il sensore di stato.

\paragraph{Attore secondario} L'attore secondario è il sistema di ticketing.

\paragraph{Precondizioni} L'attore primario è autenticato ed autorizzato dal sistema.

\paragraph{Postcondizioni} L'attore primario ha aperto il ticket di guasto e il ticket di guasto viene salvato nel sistema di ticketing

\paragraph{Scenario principale}

\begin{enumerate}
    \item L'attore primario chiede di aprire un ticket di guasto;
    \item l'attore primario specifica quali sono le motivazioni del ticket;
    \item l'attore primario specifica a quale parte del sistema sta facendo riferimento\footnote{Lampioni, sensori, ecc.};
    \item il sistema raggruppa tutte le informazioni e le invia al sistema di ticketing;
    \item il sistema di ticketing ritorna il responso sull'avvenuto salvataggio del ticket di guasto.
\end{enumerate}

\section{UC07 - Aggiunta sensore}

\paragraph{Descrizione:}
Come utente installatore/manutentore, voglio aggiungere un sensore al sistema.

\paragraph{Attore principale:}
Utente installatore/manutentore.

\paragraph{Precondizioni:}
L'utente installatore/manutentore vuole aggiungere al sistema un nuovo sensore.

\paragraph{Post-condizioni:}
Il sensore è stato aggiunto al sistema.

\paragraph{Scenario principale:}
\begin{enumerate}
    \item L'utente installatore/manutentore inserisce i dati relativi al nuovo sensore;
    \item Il nuovo sensore viene aggiunto al sistema.
\end{enumerate}

\paragraph{Scenario alternativo:}
\begin{enumerate}
    \item L'utente installatore/manutentore inserisce valori non ammissibili e il sistema ritorna un errore.
\end{enumerate}

\section{UC08 - Aggiunta area al sistema}\label{uc:08}
\paragraph{Intenzione in contesto} L'attore principale vuole aggiungere una nuova area di gestione dell'illuminazione.


\paragraph{Attore principale} L'attore principale è l'utente manutentore.



\paragraph{Precondizioni}
L'utente principale è autenticato ed autorizzato e vuole aggiungere un area di gestione illuminazione al sistema.

\paragraph{Post-condizioni}
Una nuova area di gestione è stata creata.

\paragraph{Scenario principale}
\begin{enumerate}
    \item L'attore principale crea la nuova area di gestione, inserendo i dati dell'area;
    \item L'area di gestione viene creata nel sistema.
\end{enumerate}

\section{UC09 - Invio valore luminosità}

\paragraph{Descrizione:}
Come sensore dati, voglio inviare il valore dell'intensità luminosa al sistema.

\paragraph{Attore principale:}
Sensore dati.

\paragraph{Precondizioni:}
Il sensore dati è funzionante e può leggere il valore dell'intensità luminosa.

\paragraph{Post-condizioni:}
Il sensore dati ha letto ed inviato al sistema il valore dell'intensità luminosa.

\paragraph{Scenario principale:}
\begin{enumerate}
    \item Il sensore legge il valore dell'intensità luminosa;
    \item Il sensore invia il valore letto al sistema.
\end{enumerate}

\paragraph{Scenario alternativo:}
\begin{enumerate}
    \item Il sensore non può leggere e/o inviare il valore dell'intensità luminosa.
\end{enumerate}
\section{UC10 - Segnala presenza entità}

\paragraph{Descrizione:}
Il sensore dati vuole segnalare la presenza di un'entità.

\paragraph{Attore principale:}
Sensore dati.

\paragraph{Precondizioni:}
Il sensore dati è funzionante ed è in grado di segnalare la presenza di un'entità.

\paragraph{Post-condizioni:}
Il sensore dati ha individuato e segnalato al sistema la presenza di un'entità.

\paragraph{Scenario principale:}
\begin{enumerate}
    \item Il sensore individua la presenza di un'entità nelle vicinanze;
    \item il sensore invia un segnale al sistema.
\end{enumerate}

\paragraph{Scenario alternativo:}
\begin{enumerate}
    \item Il sensore commette un'errore nell'invio della segnalazione al sistema.
\end{enumerate}
