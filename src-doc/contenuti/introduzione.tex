\chapter{Introduzione}

\section{Glossario}
Un glossario utile con alcune definizioni per lavorare al progetto al fine di chiarire eventuali termini che possono generare
dubbi interpretativi:
\begin{itemize}
    \item \textbf{Area illuminata:} sottoinsieme degli impianti di illuminazione che vengono gestiti in
modo uniforme per intensita' luminosa. Un'area illuminata e' composta da un certo numero di lampioni ed e' dotata di un sensore 
di luminosita' e di un sensore di presenza.
    \item \textbf{Lampione:} sistema di illuminazione facente parte di una specifica area illuminata provvedendo a fornire 
    l'illuminazione per una parte dell'area illuminata.
    \item \textbf{Sensore di luminosita':} sensore avente lo scopo di fornire al sistema il valore dell'intensita' luminosa 
    configurato nell'impianto d'illuminazione di riferimento al fine di valutare se apportare variazioni a tale valore.
    \item \textbf{Sensore di presenza:} sensore avente lo scopo di rilevare presenza di persone in un' area illuminata al fine 
    di variare la luminosita' dell'area a cui afferisce.
\end{itemize}