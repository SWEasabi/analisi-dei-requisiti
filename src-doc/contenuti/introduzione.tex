\chapter{Introduzione}

\section{Glossario}
Un glossario utile con alcune definizioni per lavorare al progetto al fine di chiarire eventuali termini che possono generare dubbi interpretativi:
\begin{itemize}
    \item \textbf{Area illuminata:} sottoinsieme degli impianti di illuminazione che vengono gestiti in modo uniforme per intensità luminosa. Un'area illuminata è composta da un certo numero di lampioni ed è dotata di un sensore di luminosità e di un sensore di presenza.
    \item \textbf{Lampione:} sistema di illuminazione facente parte di una specifica area illuminata provvedendo a fornire l'illuminazione per una parte dell'area illuminata.
    \item \textbf{Sensore di luminosità:} sensore avente lo scopo di fornire al sistema il valore dell'intensità luminosa configurato nell'impianto d'illuminazione di riferimento al fine di valutare se apportare variazioni a tale valore.
    \item \textbf{Sensore di presenza:} sensore avente lo scopo di rilevare presenza di persone in un'area illuminata al fine di variare la luminosità dell'area a cui afferisce.
\end{itemize}

\section{Funzionalità}
Elenco di tutte le funzionalità di cui è dotato il prodotto finale:
\begin{enumerate}
    \item Rilevamento della presenza in un'area illuminata e aumento automatico dell'intensità luminosa;
    \item Inserimento di un nuovo sensore a sistema;
    \item Aumento o riduzione manuale dell'intensità luminosa in un'area illuminata;
    \item Aumento o riduzione globale dell'intensità luminosa;
    \item Inserimento manuale di un guasto;
    \item Rilevamento automatico di un guasto;
\end{enumerate}
