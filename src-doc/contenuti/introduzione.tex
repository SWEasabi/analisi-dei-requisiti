\chapter{Introduzione}

\section{Glossario}
Un glossario utile con alcune definizioni per lavorare al progetto al fine di chiarire eventuali termini che possono generare dubbi interpretativi:

\subsection*{A}

\paragraph{Area illuminata} Sottoinsieme dei lampioni che vengono gestiti in modo uniforme per intensità luminosa. Un'area illuminata è composta da un certo numero di lampioni ed è dotata di un sensore di luminosità e di un sensore di presenza. Alcuni sinonimi: \textit{Area di gestione dell'illuminazione}.

\subsection*{L}
\paragraph{Lampione} Sistema di illuminazione facente parte di una specifica area illuminata provvedendo a fornire l'illuminazione per una parte dell'area illuminata. 

\subsection*{S}
\paragraph{Sensore di luminosità} Sensore avente lo scopo di fornire al sistema il valore dell'intensità luminosa configurato nel lampione di riferimento al fine di valutare se apportare variazioni a tale valore.

\paragraph{Sensore di presenza} Sensore avente lo scopo di rilevare presenza di persone in un'area illuminata al fine di variare la luminosità dell'area a cui afferisce.

\section{Funzionalità}
Elenco di tutte le funzionalità di cui è dotato il prodotto finale:
\begin{enumerate}
    \item Rilevamento della presenza in un'area illuminata e aumento automatico dell'intensità luminosa;
    \item Inserimento di un nuovo sensore a sistema;
    \item Aumento o riduzione manuale dell'intensità luminosa in un'area illuminata;
    \item Aumento o riduzione globale dell'intensità luminosa;
    \item Inserimento manuale di un guasto;
    \item Rilevamento automatico di un guasto;
\end{enumerate}
