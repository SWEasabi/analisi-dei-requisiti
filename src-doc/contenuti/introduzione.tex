\chapter{Introduzione}

\section{Scopo del Documento}
Nel seguente documento viene fornita una descrizione dettagliata dei requisiti del sistema o del software che deve essere sviluppato. Il documento rappresenta un punto di riferimento fondamentale per tutti gli stakeholder coinvolti nel progetto, compresi i clienti, i responsabili del prodotto, i progettisti e gli sviluppatori.

I principali obiettivi di questo documento sono i seguenti:
\begin{itemize}
    \item Comunicazione chiara dei requisiti;
    \item Base per la valutazione e l'accettazione;
    \item Guida per la progettazione e lo sviluppo;
    \item Documentazione del consenso e delle decisioni;
\end{itemize}

\section{Scopo dell'analisi dei requisiti}
L'obiettivo di SWEasabi e dell'azienda ImolaInformatica S.p.A. è lo sviluppo di un sistema per l'ottimizzazione dell'illuminazione, il prodotto presenta differenti servizi che comunicano tra loro, ogni servizio ha un compito ben preciso e si occupa di una parte del sistema per questo necessità di una architettura ben definita e strutturata.
In questo documento verranno presentate tutte le funzionalità, sia quelle che dovranno essere obbligatoriamente implementate che non, che dovrà avere il prodotto finale.

\section{Glossario}
Per evitare ambiguità relative alle terminologie utilizzate è stato creato un documento denominato \textit{Glossario}.

Questo documento contiene tutti i termini specifici di settore utilizzati nei documenti, con le relative definizioni.

\section{Maturità del documento}
Il presente documento è redatto con un approccio incrementale in modo tale da trattare modifiche o aggiunte in modo efficiente. Il documento ha raggiunto un'ottimo grado di maturità in quanto sono stati definiti tutti i requisiti, obbligatori e opzionali, richiesti dall'azienda. Il documento può essere quindi considerato definito nella sua versione attuale.

\section{Funzionalità}
Elenco di tutte le funzionalità di cui è dotato il prodotto finale:
\begin{enumerate}
    \item Rilevamento della presenza in un'area illuminata e aumento automatico dell'intensità luminosa;
    \item Inserimento di un nuovo sensore a sistema;
    \item Aumento o riduzione manuale dell'intensità luminosa in un'area illuminata;
    \item Aumento o riduzione globale dell'intensità luminosa\footnote{Questa funzionalità intende che l'utente deve poter regolare contemporaneamente l'intensità luminosa di tutti i lampioni nel sistema, vedasi sezione \ref{requisiti-funzionali}, RF\_13};
    \item Inserimento manuale di un guasto;
    \item Rilevamento automatico di un guasto;
\end{enumerate}

\section{Riferimenti}
\subsection{Riferimenti Normativi}
\begin{itemize}
    \item \href{https://github.com/SWEasabi/norme-di-progetto/releases}{Norme di progetto};
    \item \href{https://www.math.unipd.it/~tullio/IS-1/2022/Progetto/C2.pdf}{capitolato d'appalto C2}.
\end{itemize}

\subsection{Riferimenti informativi}
\begin{itemize}
    \item Slide T06 del corso di Ingegneria del Software: \\ \url{https://www.math.unipd.it/~tullio/IS-1/2022/Dispense/T06.pdf}
    \item Slide P02 del corso di Ingegneria del Software: \\ \url{https://www.math.unipd.it/~rcardin/swea/2022/Diagrammi%20Use%20Case.pdf}
\end{itemize}
