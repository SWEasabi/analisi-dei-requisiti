\section{UC08 - Gestione area illuminazione}

\paragraph{Descrizione:}
Come utente loggato, voglio poter:
\begin{itemize}
    \item inserire nel sistema nuovi impianti luminosi
    \item inserire o rimuovere impianti luminosi all'interno di un'area coperta dal servizio
\end{itemize}


\paragraph{Attore principale:}
Utente Gestore.

\paragraph{Precondizioni:}
L'utente deve essere loggato e deve essere utente gestore, dalla percentuale impostata a tempo 0.

\paragraph{Post-condizioni:}
Il sistema e/o il servizio di illuminazione hanno subito delle modifiche:
\begin{itemize}
    \item è stato aggiunto nel sistema un nuovo impianto di illuminazione 
    \item è stato aggiunto oppure rimosso un impianto di illuminazione dal servizio di illuminazione
\end{itemize}

\paragraph{Scenario principale:}
\begin{enumerate}
    \item L’utente avvia la procedura per l’inserimento di un nuovo impianto;
    \item Il sistema richiede le caratteristiche tecniche, la posizione geografica e il raggio d'azione;
    \item Il sistema richiede di specificare a quale area di illuminazione farà riferimento il nuovo impianto;
\end{enumerate}
Oppure:
\begin{enumerate}
    \item L’utente avvia la procedura per la rimozione di un impianto;
    \item Il sistema richiede di specificare a quale area di illuminazione fa riferimento l'impianto da rimuovore;
\end{enumerate}

\paragraph{Scenario alternativo:}
\begin{enumerate}
    \item L'utente fornisce un'area geografica per l'impianto inesistente o non coperta dal servizio di illuminazione scelto.
\end{enumerate}
