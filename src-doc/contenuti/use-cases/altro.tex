\section{UC02 - Regolare luminosità di un singolo lampione}

\paragraph{Descrizione:}
Come utente gestore, voglio regolare la luminosità di un singolo lampione.

\paragraph{Attore principale:}
Utente gestore.

\paragraph{Attori secondari:}
\begin{itemize}
    \item Lampadina.
\end{itemize}

\paragraph{Precondizioni:}
L'utente ha effettuato il login, è stato riconosciuto come gestore e desidera regolare la luminosità di un singolo lampione.

\paragraph{Post-condizioni:}
La luminosità del lampione è stata regolata e il cambiamento viene riflesso nella luce emanata dalla lampadina.

\paragraph{Scenario principale:}
\begin{enumerate}
    \item L'utente gestore entra nella pagina relativa al lampione;
    \item l'utente gestore modifica la luminosità del lampione;
    \item il sistema salva i cambiamenti;
    \item la lampadina modifica la propria luminosità.
\end{enumerate}

\paragraph{Scenario alternativo:}
\begin{enumerate}
    \item L'utente gestore inserisce un valore non ammesso per la luminosità e il sistema ritorna un errore.
\end{enumerate}


\subsection{UC02 - Visualizzazione del messaggio d'errore autenticazione}

\paragraph{Attore primario} L'attore primario è l'utente non autenticato.

\paragraph{Intenzione in contesto} L'attore primario desidera vedere se le credenziali che ha inserito per autenticarsi hanno problemi di qualche sorta.

\section{UC03 - regolare luminosità molteplici lampioni}

\paragraph{Descrizione:}
Come utente gestore, voglio poter manipolare lo stato di molteplici lampioni.

\paragraph{Attore principale:}

Utente Gestore.

\paragraph{Attori secondari:}
\begin{itemize}
    \item Lampadine.
\end{itemize}

\paragraph{Precondizioni:}
L'utente deve essere loggato e deve essere riconsociuto come utente gestore,
vuole modificare il livello di intensità luminosa di molteplici lampioni .


\paragraph{Post-condizioni:}
I lampioni sono manipolati rispecchiando l'intensità scelta.

\paragraph{Scenario principale:}
\begin{itemize}
    \item utente gestore accede alla pagina relativa all'insieme di lampioni.
    \item utente gestore manipola il valore relativo all'intensità luminosa.
    \item il sistema salva i cambiamenti, in linea con l'intensita' scelta.
    \item i lampioni modificano la loro intensita' luminosa in base ai parametri scelti.
\end{itemize}

\paragraph{Scenario alternativo:}
\begin{enumerate}
    \item L'utente gestore applica un valore non consentito per la luminosità , il sistema ritorna un errore.
\end{enumerate}


\section{UC04 - Apertura ticket di guasto}

\paragraph{Descrizione:}
Come \textit{attore principale}(utente gestore/sensore di guasti) desidero aprire un ticket di assistenza in caso di guasto ad un impianto di illuminazione.

\paragraph{Attori principali:}
\begin{itemize}
    \item Utente gestore;
    \item Sensore guasti.
\end{itemize}

\paragraph{Attori secondari:}
\begin{itemize}
    \item Sistema ticketing.
\end{itemize}

\paragraph{Precondizioni:}
L'utente gestore è loggato, e vuole aprire un ticket di guasto.

\paragraph{Post-condizioni:}
Il ticket di guasto è stato aperto nel sistema di ticketing.

\paragraph{Scenario principale:}
\begin{enumerate}
    \item L'attore principale rileva il guasto;
    \item l'attore principale inserisce le informazioni sul guasto;
    \item l'attore principale apre il ticket con le informazione inserite;
    \item il sistema di ticketing riceve le informazioni e conferma l'apertura del ticket.
\end{enumerate}

\paragraph{Scenario alternativo:}
\begin{itemize}
    \item L'attore principale ha inserito informazioni errate e il sistema mostra un errore;
    \item il sistema di ticketing non risponde e il sistema mostra un errore.
\end{itemize}


\section{UC05 - Creazione nuovi account}

\paragraph{Descrizione:}
Come utente installatore/manutentore, voglio creare un nuovo account.

\paragraph{Attore principale:}
Utente installatore/manutentore.

\paragraph{Precondizioni:}
L'utente installatore/manutentore è loggato, e vuole aggiungere al sistema un nuovo utente.

\paragraph{Post-condizioni:}
Il nuovo utente è stato aggiunto al sistema.

\paragraph{Scenario principale:}
\begin{enumerate}
    \item L'utente installatore/manutentore inserisce i dati personali del nuovo utente; % fare uc05.1 ?
    \item L'utente installatore/manutentore specifica il ruolo del nuovo utente;
    \item Il nuovo utente viene aggiunto al sistema.
\end{enumerate}

\paragraph{Scenario alternativo:}
\begin{enumerate}
    \item L'utente installatore/manutentore inserisce uno o più dati errati e il sistema mostra un errore.
\end{enumerate}


\section{UC06 - Aggiunta singolo lampione}

%Una breve descrizione dello use case
\paragraph{Descrizione:}
Come utente installatore/manutentore, voglio aggiungere un singolo lampione al sistema.

\paragraph{Attore principale:}
Utente installatore/manutentore.

\paragraph{Precondizioni:}
L'utente installatore/manutentore vuole aggiungere al sistema un nuovo lampione.

\paragraph{Post-condizioni:}
Il lampione è stato aggiunto al sistema.

\paragraph{Scenario principale:}
\begin{enumerate}
    \item L'utente installatore/manutentore inserisce i dati relativi al nuovo lampione;
    \item il nuovo lampione viene aggiunto al sistema.
\end{enumerate}

\paragraph{Scenario alternativo:}
\begin{enumerate}
    \item L'utente installatore/manutentore inserisce valori non ammissibili e il sistema ritorna un errore.
\end{enumerate}

\section{UC07 - Aggiunta sensore}

\paragraph{Descrizione:}
Come utente installatore/manutentore, voglio aggiungere un sensore al sistema.

\paragraph{Attore principale:}
Utente installatore/manutentore.

\paragraph{Precondizioni:}
L'utente installatore/manutentore è loggato, e vuole aggiungere al sistema un nuovo sensore.

\paragraph{Post-condizioni:}
Il sensore è stato aggiunto al sistema.

\paragraph{Scenario principale:}
\begin{enumerate}
    \item L'utente installatore/manutentore inserisce i dati relativi al nuovo sensore;
    \item Il nuovo sensore viene aggiunto al sistema.
\end{enumerate}

\paragraph{Scenario alternativo:}
\begin{enumerate}
    \item L'utente installatore/manutentore inserisce valori non ammissibili e il sistema ritorna un errore.
\end{enumerate}


\section{UC09 - Invio valore luminosità}

\paragraph{Descrizione:}
Come sensore dati, voglio inviare il valore dell'intensità luminosa al sistema.

\paragraph{Attore principale:}
Sensore dati.

\paragraph{Precondizioni:}
Il sensore dati è funzionante e può leggere il valore dell'intensità luminosa.

\paragraph{Post-condizioni:}
Il sensore dati ha letto ed inviato al sistema il valore dell'intensità luminosa.

\paragraph{Scenario principale:}
\begin{enumerate}
    \item Il sensore legge il valore dell'intensità luminosa;
    \item Il sensore invia il valore letto al sistema.
\end{enumerate}

\paragraph{Scenario alternativo:}
\begin{enumerate}
    \item Il sensore non può leggere e/o inviare il valore dell'intensità luminosa.
\end{enumerate}

\section{UC10 - Segnala presenza entità}

\paragraph{Descrizione:}
Come sensore dati voglio segnalare la presenza di un'entità.

\paragraph{Attore principale:}
Sensore dati.

\paragraph{Precondizioni:}
Il sensore dati è funzionante ed è in grado di segnalare la presenza di un'entità.

\paragraph{Post-condizioni:}
Il sensore dati ha individuato e segnalato al sistema la presenza di un'entità.

\paragraph{Scenario principale:}
\begin{enumerate}
    \item Il sensore individua la presenza di un'entità nelle vicinanze;
    \item il sensore invia un segnale al sistema.
\end{enumerate}

\paragraph{Scenario alternativo:}
\begin{enumerate}
    \item Il sensore commette un'errore nell'invio della segnalazione al sistema.
\end{enumerate}
