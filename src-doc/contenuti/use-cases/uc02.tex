\section{UC02 - Visualizzazione lista aree}

\paragraph{Attore primario} L'attore primario è l'utente gestore
\paragraph{Intenzione in contesto} L'attore primario vuole vedere una lista delle varie aree di illuminazione ed il loro stato.

\paragraph{Precondizioni} L'attore primario è riconosciuto ed autenticato dal sistema.

\paragraph{Postcondizioni} L'attore primario vdede la lista delle aree.

\paragraph{Scenario principale}

\begin{enumerate}
    \item L'utente richiede la visualizzazione della lista delle aree;
    \item il sistema fornisce la lista delle aree presenti nel sistema.
\end{enumerate}






\section{UC02 - Regolare luminosità di un singolo lampione}

\paragraph{Descrizione:}
Come utente gestore, voglio regolare la luminosità di un singolo lampione.

\paragraph{Attore principale:}
Utente gestore.

\paragraph{Attori secondari:}
\begin{itemize}
    \item Lampadina.
\end{itemize}

\paragraph{Precondizioni:}
L'utente ha effettuato il login, è stato riconosciuto come gestore e desidera regolare la luminosità di un singolo lampione.

\paragraph{Post-condizioni:}
La luminosità del lampione è stata regolata e il cambiamento viene riflesso nella luce emanata dalla lampadina.

\paragraph{Scenario principale:}
\begin{enumerate}
    \item L'utente gestore entra nella pagina relativa al lampione;
    \item l'utente gestore modifica la luminosità del lampione;
    \item il sistema salva i cambiamenti;
    \item la lampadina modifica la propria luminosità.
\end{enumerate}

\paragraph{Scenario alternativo:}
\begin{enumerate}
    \item L'utente gestore inserisce un valore non ammesso per la luminosità e il sistema ritorna un errore.
\end{enumerate}


\subsection{UC02 - Visualizzazione del messaggio d'errore autenticazione}

\paragraph{Attore primario} L'attore primario è l'utente non autenticato.

\paragraph{Intenzione in contesto} L'attore primario desidera vedere se le credenziali che ha inserito per autenticarsi hanno problemi di qualche sorta.

